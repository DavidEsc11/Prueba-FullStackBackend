Estructura Backend
Tecnologías:
Node.js: Una opción popular para el backend debido a su naturaleza basada en JavaScript y su eficiencia en operaciones de entrada/salida.
Express.js: Un marco de aplicación web de Node.js que simplifica el manejo de solicitudes HTTP y la creación de API RESTful.
Base de datos SQL o NoSQL: Dependiendo de los requisitos de la aplicación, puedes elegir entre bases de datos relacionales como PostgreSQL o bases de datos NoSQL como MongoDB.
JSON Web Tokens (JWT): Para autenticación y autorización de usuarios.
PayPal: Para procesamiento de pagos en línea.
Nodemailer: Para enviar correos electrónicos transaccionales, como confirmaciones de pedidos o restablecimiento de contraseñas.
Arquitectura:
Patrón MVC (Modelo-Vista-Controlador): Divide la aplicación en modelos (representación de datos), vistas (interfaz de usuario) y controladores (lógica de negocio y manejo de solicitudes HTTP).
Capa de servicios: Utiliza una capa de servicios para separar la lógica de negocio de los controladores y mantenerlos delgados y fácilmente mantenibles.
Middleware: Utiliza middleware para manejar funciones comunes como autenticación, autorización, manejo de errores, etc.
Organización de archivos:
- /src
  - /controllers
    - authController.js
    - productController.js
    - orderController.js
  - /services
    - authService.js
    - productService.js
    - orderService.js
  - /models
    - user.js
    - product.js
    - order.js
  - /routes
    - authRoutes.js
    - productRoutes.js
    - orderRoutes.js
  - /middlewares
    - authMiddleware.js
    - errorMiddleware.js
  - /config
    - database.js
    - jwt.js
  - app.js
Consideraciones adicionales:
Seguridad: Implementa medidas de seguridad como hash de contraseñas, validación de entrada, protección contra ataques de inyección SQL, etc.
Pruebas unitarias y de integración: Escribe pruebas para garantizar la calidad del código y la funcionalidad de la aplicación.
Escalabilidad: Diseña la aplicación teniendo en cuenta la escalabilidad para que pueda manejar un aumento en el número de usuarios y transacciones.
Monitoreo y registro: Implementa herramientas de monitoreo y registro para identificar y solucionar problemas de rendimiento y errores de manera proactiva.
